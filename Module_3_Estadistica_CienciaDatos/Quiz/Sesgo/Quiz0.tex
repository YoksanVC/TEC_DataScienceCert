%% LyX 2.3.7 created this file.  For more info, see http://www.lyx.org/.
%% Do not edit unless you really know what you are doing.
\documentclass[english]{article}
\usepackage[T1]{fontenc}
\usepackage[a4paper]{geometry}
\geometry{verbose,tmargin=3cm,bmargin=3cm,lmargin=2cm,rmargin=2cm,headheight=2cm,headsep=2cm,footskip=2cm}
\usepackage{amstext}
\usepackage{amssymb}

\makeatletter

%%%%%%%%%%%%%%%%%%%%%%%%%%%%%% LyX specific LaTeX commands.
%% Because html converters don't know tabularnewline
\providecommand{\tabularnewline}{\\}

%%%%%%%%%%%%%%%%%%%%%%%%%%%%%% User specified LaTeX commands.
\usepackage{palatino}
\pagenumbering{gobble}

\makeatother

\usepackage{babel}
\begin{document}
\begin{flushleft}
\begin{tabular}{|l|l|}
\hline 
 & \tabularnewline
\textbf{\large{}Instituto Tecnol\'{o}gico de Costa Rica} & QUIZ 0\tabularnewline
\textbf{\large{}Escuela de Computaci\'{o}n} & Entrega: Domingo 14 de Abril, a trav\'{e}s del TEC digital\tabularnewline
 & Debe subir un \emph{pdf }con la respuesta,\tabularnewline
Programa en Ciencias de Datos & generado con latex (adjunte los archivos .tex asociados).\tabularnewline
\textbf{Curso: Estadistica} & \tabularnewline
 & \tabularnewline
Profesor: Ph. D. Sa\'{u}l Calder\'{o}n Ram\'{\i}rez & Valor: 100 pts.\tabularnewline
 & Puntos Obtenidos: \_\_\_\_\_\_\_\_\_\_\_\_\tabularnewline
 & \tabularnewline
 & \tabularnewline
 & Nota: \_\_\_\_\_\_\_\_\_\_\_\_\_\_\_\_\tabularnewline
 & \tabularnewline
\cline{2-2} 
\multicolumn{2}{|c|}{}\tabularnewline
\multicolumn{2}{|l|}{Nombre del (la) estudiante: \_\_\_\_\_\_\_\_\_\_\_\_\_\_\_\_\_\_\_\_\_\_\_\_\_\_\_\_\_\_\_\_\_\_\_\_\_\_\_\_\_\_\_\_\_\_\_\_\_\_\_\_\_\_\_\_\_\_\_\_\_\_\_\_\_\_\_\_\_\_\_\_}\tabularnewline
\multicolumn{1}{|l}{} & \tabularnewline
\multicolumn{1}{|l}{Carn\'{e}: \_\_\_\_\_\_\_\_\_\_\_\_\_\_\_\_\_\_\_\_\_\_\_\_\_\_\_} & \tabularnewline
\multicolumn{1}{|l}{} & \tabularnewline
\hline 
\end{tabular}
\par\end{flushleft}
\begin{enumerate}
\item \textbf{(60 puntos)} Demuestre que el \emph{skew }o la inclinaci\'{o}n
de una funci\'{o}n de densidad exponencial:
\[
p\left(x|\lambda\right)=\lambda e^{-\lambda x}
\]
es siempre $\gamma=2$, tomando en cuenta que $\mathbb{E}\left[X^{3}\right]=\frac{6}{\lambda^{3}}$. 
\item \textbf{(40 puntos)} Con pytorch, genere 100 muestras de tama\~{n}o
$N\text{=1000}$, usando una densidad exponencial. Hagalo para dos
valores diferentes de $\lambda$ a su elecci\'{o}n. Para esas muestras,
calcule de forma vectorial el sesgo $\gamma$, y verifique la demostracion
anterior. Adjunte el archivo jupyter con tal codigo. 
\end{enumerate}

\end{document}
